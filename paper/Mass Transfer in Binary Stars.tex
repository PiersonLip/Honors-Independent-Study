% to-do
% Add more to introduction, explain more about general binaries the process of how it works
% 
% add more regarding common binaries

% Captions on all graphics

% xray accretion section

% vela x-1 properties

% maybe add xray binary type sections? to cover HMXBS and lmxbs

% good excerpt on CEE evolution over time
%As mentioned before, the common envelope evolution (CEE) has been first proposed in 1976 to account for the origin of cataclysmic variables (CVs). CVs are binaries consisting of a WD with a typical mass of around and an MS companion star with a typical mass of , in which the MS star fills its Roche lobe and undergoes the mass transfer. They typically have an orbital period of 1 to 10 h. Regarding the origin of CV systems, the question is how to form such massive WDs in such a close binary system with an orbital period of a few hours. According to the stellar evolution theory, massive WDs could be formed from cores of red giants (or supergiants) before their envelopes are removed. In this case, the orbital period of a binary system needs to be long enough to keep the red giant (or supergiant) within its Roche lobes until a massive core is formed. Then, a mechanism is needed to drastically reduce the orbital period to a few hours consequently to match the observed periods of CVs. This therefore leads to the proposal of CEE (e.g. [198], [223], [224]). By adopting the CE scenario, the origin of V 471 Tau was successfully explained [198], i.e., a detached binary in the Hyades cluster which has a WD and a K-type dwarf, with an orbital period of . This binary system is expected to evolve into a CV system within a few yr.


% Add section explaining what causes binaries to draw closer to eachother, why its needed, etc etc, 

% GwR, Mass Loss, and Magnetic braking

% things to figure out

% cause of the line in contact binaries

% distribution of masses for contact binaries

%
\documentclass[12pt, letterpaper]{article}
\usepackage{titlesec}
\usepackage{graphicx}
\usepackage{geometry}
\usepackage{abstract}
\usepackage[T1]{fontenc}
\usepackage{lipsum}
\usepackage{tabularx}
\usepackage{float}
\usepackage[sorting=none, style=nature]{biblatex} 
\usepackage[labelfont=bf, labelsep=colon, textfont=it]{caption}


\renewcommand{\abstractnamefont}{\normalfont\Large\bfseries} 

\titleformat{\section}
  {\normalfont\Large\bfseries\centering} % Formatting for the section title
  {} % Empty label (hides the number)
  {0pt} % No spacing before title
  {} % No additional formatting

\titleformat{\subsection}
    {\normalfont\Large\bfseries\centering}
    {\thesubsection}
    {.5em}
    {}
\titleformat{\subsubsection}
    {\normalfont\small\bfseries\centering}
    {\thesubsubsection}
    {.5em}
    {}


\addbibresource{sources.bib} %Import the bibliography file
\title{Mass Transfer in Binary Stars}
\author{Pierson Lipschultz\thanks{Mentored by Joseph DalSanto}}

\date{\today}
\begin{document}
\maketitle
\begin{abstract}
    \normalsize
    In this paper, I will investigate the properties of mass transfer in binary stars, including the Roche Lobe model, evolutionary stages, evolution, rate, progenitors, and resultant stars. I will use the Roche Lobe model to catalyze the stars into Detached (Wind Accretion), Roche Lobe Overflow (RLO), and Contact Binaries (CB). I will then look at stars that fit these stages and investigate them further using data from example systems corroborated with data from POSYDON.

    I found that mass transfer effected the stars by ... 
\end{abstract}

\pagebreak

\section{\centering Introduction} % amaybe include a section on x-ray emission???

    Most of the stars we see in the night sky are not actually single stars, instead consisting of multiple stars orbiting a common center of mass. A pair of such stars is called a binary. These binaries are formed in the same nature of single stars, that being through a molecular cloud molecular clouds. However, due to instability in the formation process, two stars are formed instead of one \cite{Offner_2016}. These stars orbit a common center of mass (COM)  The stars in these binary pairs will almost always a generally have a different evolutionary track then a single star, as it is incredibly likely for the two stars to interact at some point during their life \cite{Fabry_2025}. As these stars evolve and interact with each other, it will drastically affect their evolution. (see fig. \ref{fig:binary_evolution_flowchart})

    \subsection{Mass Transfer in Common Binaries} \label{MassTransferinCommon}
    While most systems do not, it is possible in systems with small enough orbital separations and a small enough mass ratio can experience a process called mass transfer (MT). Where one star (generally the larger of the two) will ``donate'' mass to the other star, called the accretor. This is incredibly likely to occur at some point during the binaries' lifespan, leading to different evolutionary outcomes then single star evolution. However, depending on the time of mass transfer, this process of mass transfer can be incredibly to detect observationally. Because of this, a large amount of studies is conducted on systems with more extreme stars, for example neutron stars(NS) or black holes(BH), this process of mass transfer leads to much more pronounced effects. This is because as the mass is transferred to the BH or NS the process leads to a large spike in X-ray emissions (sect. \ref{XrayAccretion}), which can more much more easily measured as compared the mass transfer process between a red giant (RG) a main sequence(MS) star.

    \newgeometry{left=0.75in, right=0.75in, top=0.5in, bottom=0.5in}
    \vspace*{\fill}
    \begin{figure}[H]
        \centering
        \includegraphics[width=\textwidth]{figs/Binary Evolution Flowchart.jpg}
        \caption{A comprehensive flowchart of various outcomes of binary systems in regard to masses and orbital separations. Reprint from \cite{Chen_2024}}
        \label{fig:binary_evolution_flowchart}
    \end{figure}
    \vspace*{\fill}
    \restoregeometry

    \subsection{\centering Roche Lobe Model} \label{RLModel} % maybe find the orignal rlm paper and use that to elaborate? it feels weird to not talk more here about this, however, it would be very easy to just spend forever talking about it. 
        The Roche Lobe (RL) model was proposed by Édouard Roche and defines the gravitational potential of a binary through a simple model. Simply put, it defines the region around a star where it can hold onto it mass (i.e. great enough gravitational potential). If one of the stars in the binaries' mass overflows said lobe, it will transfer mass to its binary pair. This model can be used to classify binary star populations into various populations, including \textbf{Detached Binaries} (where neither star has filled their potential), \textbf{Contact Binaries} (where both stars have filled their potentials), and \textbf{Roche Lobe Overflow}(RLO) systems (where one star has filled its potential, leading to mass transfer to an accretor).

    \vspace*{\fill}
    \begin{figure}[H]
        \centering
        \includegraphics[width=\textwidth]{Figs/RochePotential.jpg}
        \caption{ A 3D representation of the gradient of the Roche lobe}
        \label{fig:Roche_potential}
        \textit{\small Reprint from \cite{vandersluys2005}}
    \end{figure}
    \vspace*{\fill}

        \subsection{\centering Roche Lobe Overflow}\label{RLO} % maybe include equations??
        In systems where one star fully fills its RL (sect. \ref{RLModel}, fig. \ref{SemidetachedRL}) mass begins to be transferred to its binary partner. This is process is called Roche Lobe Overflow (RLO) and is defined by a donor and accretor star (sect. \ref{MassTransferinCommon}). This is incredibly likely to happen at some point in a binaries' lifespan drastically affecting the course of evolution~\cite{TaurisvandenHeuvel+2023}\cite{Chen_2024}\cite{Chen_2024}. Depending on the systems star types, masses, and eccentricity, this process of mass transfer will either be stable or unstable. When this process if unstable, it leads to either another stage called common envelope (sect. \ref{CommonEnvelope}) or a rapid merger \cite{TaurisvandenHeuvel+2023}. However, if the process of mass transfer is stable, the two stars will remain detached, slowly exchanging mass~\cite{Chen_2024} \cite{TaurisvandenHeuvel+2023}.

        In this process of mass transfer, the mass will be transferred through the Lagrangian point $L_1$, as it is the point of lowest potential between them, as seen in figure~\ref{fig:Roche_potential} \cite{TaurisvandenHeuvel+2023}. This means that it requires a velocity of efficiently 0 in order to escape the donor star. We are able to observe this very commonly in Low Mass X-ray Binaries, where a donor star transferred mass to an accretor which is a compact object (A black hole (BH) or neutron star (NS)). This process produces X-rays (sect. \ref{XrayAccretion}) which we can measure to understand the processes within the binary in greater detail.
        
       \begin{figure}[h] 
            \centering
            \includegraphics[scale = .4]{Figs/Semi-detached binary.png}
            \caption{\textit{Reprint from~\cite{TaurisvandenHeuvel+2023}}}
            \label{SemidetachedRL}
        \end{figure}
        
        I used the system V404 Cyngi with data from \cite{Bernardini_2016} and \cite{Shahbaz_1994} as an example of this behavior, as its magnitude, proximity to Earth, and large amount of studies make it an ideal choice to understand how systems like it behave. 

        \subsection{\centering Detached}\label{DetachedBinary}

        Systems where neither star fills its potential fully (see fig. \ref{DetachedBinaryRL}) are called detached binaries. Despite the fact that these systems detached in the form of their Roche Lobes, mass transfer is still possible through a processes called wind accretion (see \ref{WindAccretion}). We see this predominantly in systems called \textit{High Mass X-ray Binaries}, where a supergiant star transfers mass to a compact object via wind accretion. This process leads to an increase in X-ray Emission (sect. \ref{XrayAccretion}), which we can easily measure~\cite{TaurisvandenHeuvel+2023}. It is important to note that these systems are not experiencing full-blown RLO (sect\ref{RLO}), however, they tend to be incredibly close to doing so~\cite{TaurisvandenHeuvel+2023}. It is important to note that these systems transfer mass through both wind accretion, atmospheric Roche Lobe Overflow, and full-blown RLO.

        I used the system Vela X-1 \cite{Kretschmar_2021} as an example of this, as it is generally regarded as the ``archetypical wind accretor''~\cite{Kretschmar_2021} 

        \subsubsection{Wind Accretion Mass Transfer} \label{WindAccretion}
        Wind accretion very different from normal mass transfer in a binary. All stars produce `wind', i.e. mass which is pushed away from the star. This process is called stellar winds, occurring when various types of mass is ejected at speed from the star. All stars have different processes of wind, with some have very high velocities of the wind, and others have lower velocities.~\cite{Lamers_1999} In binary stars this process allows mass to be transferred from a donor to accretor.
        
        
        \begin{figure}[H]
            \centering
            \includegraphics[scale = .4]{Figs/Detached binary.png}\\
            \textit{Reprint from~\cite{TaurisvandenHeuvel+2023}}
            \caption{Graphic of the RL model with regard to how much potential is being filled in a detached binary. Note here that has both a `top-down' perspective and from the side.}
            \label{DetachedBinaryRL}
        \end{figure}
        

        \subsection{\centering Contact Binary}
        In binaries where RLO occurs it is possible for the donor star to fill up not just its potential, but the accretors' as well. In cases like these the star then begins to fill up the binaries itself potential.\ (i.e.\ the area between the two ridges~\ref{ContactBinaryRL}). This process generally is not stable, as most stars this in stage generally are experiencing a brief stage of their evolution called common envelope (CE). (See section~\ref{CommonEnvelope}) However, in cases where it is stable, the star will continue to evolve as one body (sect. \ref{CommonEnvelopeStableEvoluton}).

        
        \begin{figure}
            \centering
            \includegraphics[scale = .4]{Figs/Conact Binary.png}

            \caption{\textit{Reprint from~\cite{TaurisvandenHeuvel+2023}}}
            \label{ContactBinaryRL}
        \end{figure}
        
        \subsubsection{Common Envelope}\label{CommonEnvelope}
            The CE stage of binary evolution has two distinct outcomes, both of which heavily depend on the systems initial conditions. The two stars can either merge, in most cases becoming a single star, or that one star can be ejected, reverting the system back to a detached state (sect. \ref{DetachedBinary}) In this process the potential of the systems is completely filled, at which point mass stars being pushed out of the entire system through $L_2$. This mass then begins to form a disk around the stars, orbiting at a reduced rate compared to the stars~\cite{TaurisvandenHeuvel+2023}. This mass that has been ejected from the system entirely will create ionized gas around the stars, which can be observed from Earth. ~\cite{TaurisvandenHeuvel+2023}

        \subsubsection{Stable Evolution}\label{CommonEnvelopeStableEvoluton}
            In systems which are stable the stars will share mass and that their shells will evolve in tandem, this process is called ``homogenous chemical evolution''. (See fig~\ref{fig:binary_evolution_flowchart} \cite{Chen_2024}). As these stars transfer mass, their mass ratio ($q$) will oscillate around a value of $q=1$, eventually reaching $q_{min}$ at which point the stars will rapidly merge (see fig. \ref{qEvolution}). \cite{Pešta_2023}
            
        \begin{figure}[H]
            \centering
            \includegraphics[scale = .3]{figs/reused-figs/q-ratio_evolution_farby.png}
            \caption{Reprinted from \cite{Fabry_2025}. These graphs show the effects of modeling energy transfer(ET) on the systems' evolution.}
            \label{qEvolution}
        \end{figure}

        
        I used W Ursae Majoris (W UMa) as an example of these systems, as it is used an example system in order to categorize these contact binaries as a whole.
    \subsection{X-ray Binaries} 
        \subsubsection{X-rays caused by accretion onto a compact object} \label{XrayAccretion}
            In 1962 the first X-ray binary was discovered by Riccardo Giacconi and colleagues. This system, Scorpius X-1, is so bright in x-ray that it actively raises ionization levels in Earths atmosphere when above the horizon. \cite{TaurisvandenHeuvel+2023} \cite{Giacconi_1962} In the years following, it was discovered that that Scorpius X-1 consisted of a normal star and neutron star. Since then, many thousands more have been discovered\cite{Haardt_1993}

            These x-rays are produced from the friction of in-spiralling matter against itself, which causes it to become incredibly hot, with the inner disk reaching temperatures of $\geq$ 10 million K, causing it to emit a large amount of x-rays. In systems with a NS accretor, the surface of the NS itself will also emit a large amount of x-rays \cite{TaurisvandenHeuvel+2023} (fig. \ref{XrayAccretionMarkVis}). 

            \begin{figure} [H]
                \centering
                \includegraphics[width=\textwidth]{figs/reused-figs/markGarlic-Xrayaccretion.png}
                \caption{Reprinted from \cite{TaurisvandenHeuvel+2023}, original work by Mark Garlick, \copyright Mark Garlick. Here we see matter from a normal star falling onto a compact object, with the compact object being the source of x-rays}
                \label{XrayAccretionMarkVis}
            \end{figure}
        % \subsubsection{Low Mass X-ray Binaries (LMXBS)} \label{lmxbs}
        % \subsubsection{High Mass X-ray Binaries (HMXBS)} \label{HMXBS}
        % \subsection{\centering X-ray Binaries}
        
\section{\centering Data Acquisition}
    \subsection{\centering Vela X-1 (Detached)}
    
    \begin{table}
            \subsubsection{Known properties}
            \begin{center}
                \begin{tabular}{||c | c  c||} 
                 \hline
                 & Vela X-1 A & Vela X-1 B  \\ 
                 \hline\hline
                 \textbf{Star Type} & Neutron Star & Supergiant \cite{Kretschmar_2021} \\ 
                 \hline
                 \textbf{Masses}\(M_\odot\) & $\ge$ 1.8 \cite{Kretschmar_2021} & 20–30 \cite{Kretschmar_2021} \\
                 \hline
                 \textbf{Radius} & 11-12.5$_{KM}$ \cite{Kretschmar_2021} & 30 \(R_\odot\)
                 \cite{Kretschmar_2021} \\ % note that the stars are not spherical
                 \hline 
                 \textbf{Separation} &  \multicolumn{2}{c||}{2$kpc$ \cite{Kretschmar_2021}} \\
                 \hline 
                 \textbf{Mass Loss Rate} & \multicolumn{2}{c||}{$10^{-6} M_\odot yr^{-1}$ \cite{Kretschmar_2021}} \\
                 \hline
                 \textbf{Eccentricity} & \multicolumn{2}{c||}{$ e \approx  0.0898)$ \cite{Kretschmar_2021}} \\
                 \hline
                \end{tabular}
                \caption{Properties of Vela X-1} 
                \label{VelaX1} 
            \end{center}
    \end{table}

        Vela X-1 consists of a Neutron Star and Supergiant and is an Eclipsing and pulsing HMXB (sect. \ref{HMXBS}). This means that the Neutron star passed behind the Supergiant every 8.94 days \cite{Falanga_2015}, leading to a variable luminosity between $10^{36}$ $erg$ $s^{-1}$ and $10^{37}$ $erg$ $s^{-1}$. Additionally, the neutron star itself is spinning every 293 seconds. \cite{Kretschmar_2021}. 
        
        Vela X-1 is described as an archetypical wind accretor, as it is a system which is undergoing wind accretion in a stable, predictable, and easy to measure way. The x-ray emission is persistent as well its broadband spectra. Astronomers use Vela X-1 as examples when looking at other systems with comparable x-ray emissions. \cite{Kretschmar_2021}
        
        The wind accretion (see \ref{WindAccretion}) comes in the form of wind from supergiant star (Vela X-1B) falling onto the neutron star. This wind does not have a very high velocity, but because the supergiant has almost filled it RL \cite{Kretschmar_2021}, the wind mass can easily escape, falling onto the NS. This accretion process (Sect. \ref{XrayAccretion}) is what creates the prominent X-ray emission.

        \subsubsection{Notes}   
            Vela X-1A has a mass defined to be higher than the Chandrasekar limit, which allows it to be used to help create models for compact stars and equations-of-state. \cite{Kretschmar_2021}
        
    \subsection{\centering W Ursae Majoris (Contact Binary)}
        \subsubsection{Known properties}

        \begin{table} 

            \begin{center}
                \begin{tabular}{||c | c c||} 
                    \hline
                    & W UMa A & W UMa B \\ 
                    \hline\hline
                    \textbf{Masses}\(M_\odot\) & 1.139 ± 0.019\cite{Gazeas_2021} & 0.551 ± 0.006\cite{Gazeas_2021} \\
                    \hline
                    \textbf{Radius}\(R_\odot\) & 1.092 ± 0.016\cite{Gazeas_2021} & 0.792 ± 0.015\cite{Gazeas_2021} \\
                    \hline
                    \textbf{Temperature}$K$ & 6450 ± 100 \cite{Gazeas_2021}  & 6170 ± 21 \cite{Gazeas_2021} \\
                    \hline
                    \textbf{Luminosity}\(L_\odot\) & 1.557 ± 0.166\cite{Gazeas_2021} & 0.978 ± 0.071\cite{Gazeas_2021}   \\ 
                    \hline
                    \textbf{Distance} & \multicolumn{2}{c||}{52$pc$ \cite{GaiaCollab_2018}}\\
                    \hline
                    \textbf{Max Magnitude} & \multicolumn{2}{c||}{7.75 \cite{Malkov_2006}} \\
                    \hline
                    \textbf{Min Magnitude} & \multicolumn{2}{c||}{8.48 \cite{Malkov_2006}} \\
                    \hline
                    \textbf{Period} & \multicolumn{2}{c||}{.3336 $days$ \cite{Gazeas_2021}}\\
                    \hline
                    \textbf{Inclination Plane}  & \multicolumn{2}{c ||}{$88.4 \pm 0.8^\circ$ \cite{Gazeas_2021}} \\
                    \hline
                \end{tabular}
                \caption{Properties of W Ursae Majoris} 
                \label{WUmaTable} 
            \end{center}
        \end{table}

        W UMa is a contact binary, meaning that the two stars are physically `connected' by their mass. This system is known as an archetype because is it has a high magnitude at 7.75 at peak and 8.48 at minimum (table \ref{WUmaTable}), meaning that its fairly easy to observe the variability. We can measure said variability in the form of light curves, which reveal a distinct nature different which is different from non-contact binaries (fig. \ref{WUMaLightcruve}). This magnitude variability is due to the fact that the binary is eclipsing due to its low inclination plane (table \ref{WUmaTable}), meaning that one of the stars will pass behind the other in its orbit relative to the Earth. 

        Because of the prominent nature of this binary, similar contact binaries are called referred to as `UWMa type' if they also possess said eclipsing nature. 

        \begin{figure}[H]
            \centering
            \includegraphics[scale= .3]{figs/W Uma Lightcurve.png}
            \caption{\textit{Reprint from~\cite{Morgan_1997}}}
            \label{WUMaLightcruve}
        \end{figure}

        % \subsubsection{Simulated}
        % Simulated data is good because of xyz and is useful because of xyz. data was made using xyz

    \subsection{\centering Roche Lobe overflow in V404 Cyngi}
        \subsubsection{Known properties}
        \begin{table}
            \begin{center} 
                \begin{tabular}{||c | c c||} 
                 \hline
                 & \textbf{V404 Cyngi B (Donor)} & \textbf{V404 Cyngi A (Black Hole)} \\ 
                 \hline\hline
                 \textbf{Star Type} & Early K-type Giant & Black Hole \\ 
                 \hline
                 \textbf{Masses}& $.7_{M_\odot}$ \cite{Bernardini_2016} & $9_{M_\odot}$ \cite{Shahbaz_1994} \\
                 \hline
                 \textbf{Radius} & $6.0_{R_\odot}$ \cite{Shahbaz_1994} &  \\
                 \hline
                 \textbf{Temperature} & $4800_K$ \cite{Shahbaz_1994} & \\
                 \hline
                 \textbf{Luminosity} & $10.2_{L_\odot}$ \cite{Shahbaz_1994} &  \\ 
                 \hline
                 \textbf{Distance} & \multicolumn{2}{c||}{$2390_{pc}$ \cite{Bernardini_2016}} \\
                 \hline
            \end{tabular}
            \caption{Properties of V404 Cygni} 
            \label{V404Data} 
            \end{center}
        \end{table}

        V404 is a LMXB (sect \ref{lmxbs}), meaning that the donor star has a relatively low mass. In this system the material being accreted by V404 Cyngi A forms an accretion disk, greatly increasing the luminosity of the system.

    \subsection{POSYDON Simulations}
         I used data generated by POSYDON \cite{Fragos_2023} in corroboration with three observed systems in order to fully understand the depth of the process of mass transfer in Binary Systems. This is because while catalogues of contact binaries, HMXBS and LMXBs exist, there is not enough of them to get a true grasp of the full picture. Hence, I used POSYDON. This dataset was simulated on the NU Super computing Cluster, QUEST. POSYDON is developed and maintained by a team of astrophysicists and computer scientists working at the Université de Genève and Northwestern University. POSYDON uses an additional script called MESA, which is dedicated to single star and binary evolution. POSYDON utilities MESA on a much larger scale in order to simulate full populations. The data was stored in the form of a .h5 file, containing a total of $\approx$ 6.1 million rows and 83 columns. (See greatly reduced example of the data frame in (table \ref{POSYDONDataExample}) and an HR Diagram of the full dataset in Fig. \ref{EntireDataSetHR})

         \begin{table}
            \centering\
            \footnotesize
            \begin{tabularx}{\textwidth}{||X | X | X | X | X | X | X | X ||}
                \hline
                \textbf{Binary ID} & 
                \textbf{System State} & 
                \textbf{Orbital Period (days)} & 
                \boldmath$\log_{10}$ \textbf{Mass Transfer Rate} & 
                \textbf{Donor State} & 
                \textbf{Donor Mass} $M_\odot$ & 
                \textbf{Accretor State} & 
                \textbf{Accretor Mass} $M_\odot$
                \\ \hline
                $54$ & Detached & $0.047520$ & $-99.00000$ & NS & $1.196033$ & stripped He Core He burning & $\approx 1.002$ \\
                \hline
                $183$ & Detached & $0.0429883$ & $ \approx -80.8$ & NS & $1.196033$ & stripped He Core He burning & $\approx .9957$ \\
                \hline
            \end{tabularx}
            \caption{Example of POSYDON data, heavily modified for readability}
            \label{POSYDONDataExample}
        \end{table}

        \begin{figure} [H]
            \centering
            \includegraphics[width = \textwidth]{figs/EntireDataSetHR.png}
            \caption{HR Diagram of the donor star for the full POSYDON dataset. Note that the color of the plotted points correspond to the $log_{10}$ mass of the donor star. Generated with Matplotlib}
            \label{EntireDataSetHR}
        \end{figure}
        \subsubsection{Data Processing}
            With my research I sought to contextualize these very specific systems, allowing one to better understand how these systems play into the larger picture of binaries. In order to do this I utilized a large dataset generated from POSYDON. In order to properly analyze this data I utilized Python with a large quantity of packages. These packages included Pandas \cite{reback2020pandas}, Matplotlib \cite{Matplotlib}, and NumPy \cite{harris2020array}. These tools allowed me to rapidly and efficiently analyze the large amount of data, something this paper would not have been possible without.

    \section{\centering Results}
        Through my research I found a number of interesting discoveries, both regarding the populations of binaries as a whole, but also some applicable to systems themselves.
  
        \subsection{\centering Roche Lobe Overflow}
        Full-blown RLO occurs in both all types of XrB's, including HMXBS and LMXBs, however, it is more common in LMXBS.

        \begin{figure}[H]
            \centering
            \includegraphics[scale = .6]{figs/Generated Figs/X-ray Binaries Eccentricty Distribution.png}
            \caption{Eccentricity distribution of X-ray Binaries}
            \label{XrayBinaryEccenDistro}
        \end{figure}

        In figure \ref{XrayBinaryEccenDistro} we can see that x-ray binaries have a standard Gaussian distribution, with a peak at around $\approx .24$  

        \begin{figure}[H]
            \centering
            \includegraphics[scale = .6]{figs/Generated Figs/X-ray Binaries Star Two Mass Distribution.png}
            \caption{Mass distribution of X-ray Binaries}
            \label{XrayBinaryMassDistro}
        \end{figure}
        In figure \ref{XrayBinaryMassDistro} we see that the majority of X-ray binaries masses are centered around $\approx 5.5$

        \begin{figure}[H]
            \centering
            \includegraphics[scale = .6]{figs/Generated Figs/ X-ray Binaries Star Two Mass log10 F star radius T.png}
            \caption{HR diagram of the donor star in X-ray Binaries}
            \label{XrayBinaryHRDiagram}
        \end{figure}
        In figure \ref{XrayBinaryHRDiagram} we can see that X-ray binaries' donor stars generally follow a semi-standard HR diagram 
        
        \subsubsection{V404 Cygni Results}
            \begin{figure}[H] 
                \centering
                \includegraphics[scale = .6]{figs/Generated Figs/ V404 Cyngi HR Context Star Two Mass log10 F star radius 5.png}
                \caption{HR Diagram with reference for V404 Cyngi}
                \label{V404Context}
            \end{figure}

            I discovered that does not fall onto the two main simulated evolutionary tracks for LMXBs. There are a lot of different reasons why this could be, including mass transfer limiting the luminosity, error, etc. 
            
            %Add in a graph of a full HR Diagram and show the area of overlap, propose actual reason why

        %contact binaries
        \subsection{\centering Contact Binaries}
            Through my analysis I discovered that contact binaries had some of the most interesting results. 
        \begin{figure}[H]
            \centering
            \includegraphics[scale = .6]{figs/Generated Figs/Contact binaries Star Two Mass Distribution.png}
            \caption{Star two mass distribution for contact Binaries}
            \label{contactBinaryStar2MassDistro}
        \end{figure}

        In figure \ref{contactBinaryStar2MassDistro} we can see that contact binaries mass distribution feature a prominent spike at around one solar mass, with a distribution centered around 10, and then a scattered amount afterward. 

        \begin{figure}[H]
            \centering
            \includegraphics[scale = .6]{figs/Generated Figs/Contact binaries Star Two Mass Distribution.png}
            \caption{Star one mass distribution of contact Binaries}
            \label{contactBinarStar1MassDistro}
        \end{figure}

        Note how similar this is to the star two distribution. From the POSYDON data, I found the mean mass of star one in contact binaries to be $\approx 10.703$ and star two to be $\approx 10.6548$. This is congruent with what previous papers have found (\cite{Fabry_2025}). This is due to the nature of mass transfer in contact binaries leading to a stabilization in mass. As a contact system evolves, the mass transfer causes q (the mass ratio between stars) to stabilize to a value of $q=1$ \cite{Fabry_2025}. We can clearly see this oscillation and then stabilization in \textbf{figure \ref{qEvolution}}.

        \begin{figure}[H]
            \centering
            \includegraphics[scale = .6]{figs/Generated Figs/ Contact binaries Star Two Mass log10 F star radius T.png}
            \caption{HR diagram of contact Binaries}
            \label{contactBinaryHRDiagram}
        \end{figure}

        This is one of the more interesting results, as we can see that contact binaries form a very specific population on the HR diagram, falling on specific linear with a linear relationship. I believe this is because of the stabilizing natures regarding mass ratios in contact binaries. It is important to note that these values are much above what we observe in contact binaries. Again, there are many reasons this could be, including the inherent skewing of the grid, the nature of observing contact binaries, and more... \textit{adding more here.} 


        
        \subsubsection{W UMa Cygni Results}
            \begin{figure}[H]
                \centering
                \includegraphics[scale = .6]{figs/Generated Figs/ W Ursae Majories HR Context Star Two Mass log10 F star radius 5.png}
                \caption{HR Diagram with reference for W UMa using W UMa A}
                \label{WUMaResults}
            \end{figure}
            In figure \ref{WUMaResults} we see that W UMa also follows this linear relationship, however, it has a much lower temperature and luminosity then most of the population, but in line with other observed contact binaries.  

        \subsection{\centering Detached Binaries}
            \begin{figure}[H] 
                \centering
                \includegraphics[scale = .6]{figs/Generated Figs/ High Mass X-ray Binaries Star Two Mass log10 F star radius T.png}
                \caption{HR Diagram of HMXBS}
                \label{DetachedBinaryHRDiagram}
            \end{figure}

            \subsubsection{Vela X-1 Results}
            \begin{figure}[H] 
                \centering
                \includegraphics[scale = .6]{figs/Generated Figs/ Vela X-1 HR Context Star Two Mass log10 F star radius 5.png}
                \caption{HR Diagram with reference for Vela X-1}
                \label{VelaX1Results}
            \end{figure}
            In figure \ref{VelaX1Results} we can see that Vela X-1 Falls into a very normal location for main sequence stars.

\section{\centering Conclusion}
    In conclusion mass transfer heavily affects the evolution of binary stars, leading them to evolve in a uniquely different way then to single star evolution. 

\section{\centering Discussion}   
    Originally I planned on simulating grids for all the types of systems, however, due to time constraints I chose to only focus on one type. 

    \subsection{Git Galore}
    Originally, I started working on this project using Overleaf, however, due the amount of graphs, the time to compile started to rapidly climb up, and eventually I just decided to switch to compiling it locally. On-top of this, I figured I would set up a GitHub in order to additionally push the code and graphs as well. This turned out to absolutely be the right decision, giving me way more freedom.
    
\printbibliography[
heading=bibintoc,
title={\centering Sources}
]

\end{document}