\documentclass[12pt, a4paper]{article}
\usepackage{titlesec}
\usepackage{graphicx}
\usepackage{geometry}
\usepackage{abstract}
\usepackage[T1]{fontenc}
\usepackage{lipsum}
\usepackage{tabularx}
\usepackage{float}
\usepackage{amssymb}
\usepackage[sorting=none, style=nature]{biblatex} 
\usepackage[labelfont=bf, labelsep=colon, textfont=it]{caption}

\addbibresource{sources.bib} %Import the bibliography file
\title{Mass Transfer in Binary Stars Research Essay}
\author{Pierson Lipschultz}
\sloppy
\begin{document}
\date{\today}
\maketitle

\begin{abstract}

\end{abstract}

\section{Summary}
    In this paper I investigated the effects of mass transfer in binary star systems. In order to do this,  I used data from three observed systems (V404 Cygni, Vela X-1, and W Ursae Majoris) in conjunction with a large (one million simulated stars) private dataset which I obtained from Northwestern CIERA (\url{https://ciera.northwestern.edu/}). This data was simulated on the NU high performance computing cluster \textbf{Quest}. 

    I first used the Roche Lobe model to categorize the systems into specific stages, allowing me to investigate the systems to a greater depth. To analyze the data I used Python with a multitude of packages (NumPy, Matplotlib, Bokeh, Pandas). As I analyzed the data, I found there were certain actions regarding data processing and graphing that I was doing repeatedly, so I wrote a custom Python script in order to streamline the process. This script allowed me to change minor things in the graphs (like the title of color bars) quickly and apply them to all the curated graphs. 

	Additionally, I wrote the entire paper in LaTeX (\url{https://www.latex-project.org/}), a typesetting system which is the standard for high level academic papers. I challenged myself to learn and use this fromat in order to prepare myself for further academia. Using LaTeX also allowed the graphs being generated by my Python scripts to automatically be imported and/or updated into the paper, saving large quantities of time.
	The entire paper, all of my data, code, and figures, can all be found on GitHub \url{https://github.com/PiersonLip/Honors-Independent-Study}.



\section{Research Question}
    My research question was `How does mass transfer affect the evolution of binary stars?' This question was prompted by a previous investigation in another population of binary star systems called ultra-compact x-ray binaries. While I was investigating these, they had a unique reaction to mass transfer (a large part of what made them a population). I then became curious how mass transfer affected populations of binaries as a larger whole and decided to further investigate it.
    Discuss how your research question may have changed as you learned more about your topic.
    As I learned more about mass transfer I found that a lot of the states of the Roche Lobe model were temporary, something that the star would evolve into and out of. This did not change the research question itself, but changed the way I was investigating the topic, causing me to look more at the population of stars themselves instead of the current state of the observed systems. I.e. greatly widening the time span I was looking at.
\section{Sources}
    \textit{Describe the types of sources that you used to answer your question.  Were library databases, books, websites, or other sources appropriate for your topic?  Why or why not?}\\
    I used ADS \url{https://ui.adsabs.harvard.edu/} and ArXiv \url{https://arxiv.org}/ in order to find recent and up-to-date studies regarding my systems. Additionally, I used the textbook Physics of Binary Star Evolution \url{https://press.princeton.edu/books/hardcover/9780691179070}, as it provided the perfect balance of an overview which was also incredibly well cited, allowing me to delve deeper into the topics which I was interested in.
\section{Research strategies}
    \textit{Think: how did your research strategies change throughout the research process, including where you looked for information as you moved from a beginner to an expert on the topic.}\\
    As I began to understand the topic more, I started using Python to directly analyze data and apply it to my results. Additionally, I started by using the textbook, which allowed me to get a good grounded understanding, but when I started investigating very specific properties I then began to use academic papers.
\section{Information Evaluation}
    \textit{Evaluate and reflect: how did you evaluate the information you found?  Did you encounter sources that did not provide good information or even provided biased information?  What expert sources did you use?}\\
    I evaluated my information through a multitude of ways. The first thing I always looked at was the publishing date. There were plenty of papers which would provide a great foundations of info on a binary system (for example \url{doi.org/10.1093/mnras/271.1.L10}, which I used for V404 Cygni), but due to their age the actual measurements themselves were out of date. In order to make sure the data I was using was as up-to-date as possible, I would use these papers to lay a groundwork, and then reinforce it with a newer paper with more accurate measurements (for example, \url{https://doi.org/10.1051/0004-6361/202346571} which had the measurements I used for V404 Cygni)

    I used only expert sources in order to make sure my data and explanations were as accurate as possible.

\section{Information Gathering and Methodology}
    \textit{What information was gathered to create new knowledge?  Explain your methodology.}\\
    I used data generated by POSYDON in corroboration with three observed systems in order to fully understand the depth of the process of mass transfer in Binary Systems. This is because while catalogues of contact binaries, HMXBS and LMXBs exist, there is not enough of them to get a true grasp of the full picture. Hence, I used POSYDON. This dataset was simulated on the NU High Performance Computing Cluster, QUEST. POSYDON is developed and maintained by a team of astrophysicists and computer scientists working at the Université de Genève and Northwestern University. POSYDON uses an additional script called MESA, which is dedicated to single star and binary evolution. POSYDON utilities MESA on a much larger scale in order to simulate full populations. The data was stored in the form of a .h5 file, containing a total of $\approx$ 6.1 million rows and 83 columns.

    This data was provided in a totally “raw” format, containing the entire database. This allowed me to 
\section{Results}
\textit{    Explain the results of the study and what you learned.
}\section{What Would I change}
    What would you change about your research process if you were to do it over again?
\end{document}